\documentclass[a4paper,12pt]{article}
\usepackage{amsmath,amssymb,amsfonts,amsthm}
\usepackage{tikz}
\usepackage [utf8x] {inputenc}
\usepackage [T2A] {fontenc} 
\usepackage[russian]{babel}
\usepackage{cmap} 

% Так ссылки в PDF будут активны
\usepackage[unicode]{hyperref}

% вы сможете вставлять картинки командой \includegraphics[width=0.7\textwidth]{ИМЯ ФАЙЛА}
% получается подключать, как минимум, файлы .pdf, .jpg, .png.
\usepackage{graphicx}
% Если вы хотите явно указать поля:
\usepackage[margin=1in]{geometry}
% Или если вы хотите задать поля менее явно (чем больше DIV, тем больше места под текст):
% \usepackage[DIV=10]{typearea}

\usepackage{fancyhdr}

\newcommand{\bbR}{\mathbb R}%теперь вместо длинной команды \mathbb R (множество вещественных чисел) можно писать короткую запись \bbR. Вместо \bbR вы можете вписать любую строчку букв, которая начинается с '\'.
\newcommand{\eps}{\varepsilon}
\newcommand{\bbN}{\mathbb N}
\newcommand{\dif}{\mathrm{d}}

\newtheorem{Def}{Определение}


\pagestyle{fancy}
\makeatletter % сделать "@" "буквой", а не "спецсимволом" - можно использовать "служебные" команды, содержащие @ в названии
\fancyhead[L]{\footnotesize Электричество и магнетизм}%Это будет написано вверху страницы слева
\fancyhead[R]{\footnotesize ФМХФ МФТИ}
\fancyfoot[L]{\footnotesize \@author}%имя автора будет написано внизу страницы слева
\fancyfoot[R]{\thepage}%номер страницы —- внизу справа
\fancyfoot[C]{}%по центру внизу страницы пусто

\renewcommand{\maketitle}{%
	\noindent{\bfseries\scshape\large\@title\ \mdseries\upshape}\par
	\noindent {\large\itshape\@author}
	\vskip 2ex}
\makeatother
\def\dd#1#2{\frac{\partial#1}{\partial#2}}


\title{3.1.1 \\ Магнитометр}
\author{Егор Берсенев} 
\date{15 апреля 2016 г.}

\begin{document}
	\maketitle
	\section{Цель работы}
		 Определить горизонтальную состовляющую магнитного поля Земли и установить количественное соотношение между единицами электрического тока в системах СИ и СГС.
	\section{Оборудование}
		Магнитометр, осветитель со шкалой, источник питания, вольтметр, электромагнитный переключатель, конденсатор, намагниченный стержень, прибор для определения периода крутильных колебаний, секундомер, рулетка, штангенциркуль.
	\section{Теоретическая часть}
		Магнитометр состоит из нескольких последовательно соединенных круговых витков К, расположенных вертикально. В центре кольца К на тонкой неупругой вертикальной нити подвешена короткая магнитная стрелка С. Жестко связаная со стрелкой крыльчатка погружена в масло для демпфирования колебаний. В отсутствие других магнитных полей, стрелка располагается по направлению горизонтальной составляющей магнитного поля земли $\mathbf{B_0}$. При появлении дополнительного горизонтального поля $\mathbf{B_\perp}$ стрелка С установится по равнодействующей $\mathbf{B_\Sigma}$. Дополнительное поле может быть создано либо магнитным стержнем на кольце, либо током через кольцо. Поле можно считать однородным, т.к. размеры стрелки много меньше радиуса кольца. Поле стержня на перпендикуляре к нему:
		\begin{equation}
			B_1 = \frac{\mu_0}{4\pi}\frac{\mathfrak{M}}{R^3}
		\end{equation}
		Поле в центре кольца с током по закону Био и Савара:
		\begin{equation}
			B_2 = \frac{\mu_0I}{2R}N
		\end{equation}
		Измерив угол отклонения стрелки $\phi$, можно связать поля $B_0$ и $B_\perp$:
		\begin{equation}
			B_\perp = B_0 \tan \phi
		\end{equation}
		\subsection{Определение горизонтальной составляющей магнитного поля Земли}
		Для определения горизонтальной составляющей магнитного поля Земли возьмем тонкий магнитный стержень и установим в отверстие на горизонтальном диаметре. Измерив тангенс угла отклонения стрелки
		\begin{equation}
			\tan \phi_1 = \frac{x}{2L}	 
		\end{equation}
		можно рассчитать поле $B_0$, если исключить магнитный момент стержня. Для этого измерим период крутильных колебаний стержня в магнитном поле Земли. Тогда:
		\begin{equation*}
			M_{\text{мех}} = \mathfrak{M}B_0\sin\alpha \simeq \mathfrak{M}B_0\alpha
		\end{equation*}
		Запишем уравнение движения стержня:
		\begin{equation*}
			J\ddot{\alpha}+\mathfrak{M}B_0\alpha = 0
		\end{equation*}
		Выразим отсюда период колебаний:
		\begin{equation}
			T = 2\pi\sqrt{\frac{J}{\mathfrak{M}B_0}}
		\end{equation}
		Момент инерции цилиндрического стержня:
		\begin{equation}
			J = m\left(\frac{l^2}{12}+\frac{r^2}{4}\right)
		\end{equation}
		Объединив все формулы, получим:
		\begin{equation}
			B_0 = \frac{2\pi}{TR}\sqrt{\frac{\mu_0JL}{2\pi Rx_1}}
		\end{equation}
		\subsection{Определение электродинамической постоянной}
		Для определения электродинамической постоянной необходимо провести измерения одного и того же тока в разных системах: $I_{\text{СИ}}$ и $I_{\text{СГС}}$.
		\begin{equation}
			c = 10\frac{I_{\text{СГС}}}{I_{\text{СИ}}}
		\end{equation}
		Пропуская ток через витки магнитометра, измерем тангенс угла отклонения стрелки и по формулам (2) и (3) рассчитываем величину:
		\begin{equation}
			I_{\text{СИ}} = \frac{2B_0R}{\mu_0N}\tan\phi_2 = A\tan\phi_2
		\end{equation}
		Величина A является постоянной прибора.
		В случае, если $B_0$ известно, то определение силы тока не требует сравнения с эталонами. Одновременно тот же ток измеряется в системе СГС. Если разрядить конденсатор емкости $C$, заряженный до напряжения $U$, то через витки протечет заряд $Q=CU$. Если делать это $n$ раз в секунду, то средний ток через витки будет равен:
		\begin{equation}
			I_{\text{СГС}} = CUn
		\end{equation}
		Таким образом, измерение тока в СГС сводится к нахождению двух, абсолютным образом определенных величин.
		Для проведения опыта возьмем конденсатор, емкость которого выражена в сантиметрах, а измерения напряжения проведем вольтметром, прокалиброванным в СИ.
		\section{Ход работы}
		\subsection{Определение горизонтальной составляющей магнитного поля Земли}
		Измерим параметры установки
		\begin{center}
			\begin{tabular}{c|c|c}
				   $L$ & 104.5 & см \\ 
			\hline $\sigma L$ & 0,5 & см \\ 
			\hline $l$ & 40 & мм \\ 
			\hline $\sigma l$ & 0.05 & мм \\ 
			\hline $d$ & 5 & мм \\ 
			\hline $\sigma d$ & 0.05 & мм \\ 
			\hline $m$ & 5.9 & гр \\
			\hline $R$ & 20 & см \\
		\end{tabular}
		\end{center} 
		Проведем измерения отклонения кольца:
		\begin{center}
			\begin{tabular}{c|c}
					   $x_{1+}$, см & $x_{1-}$, см \\
				\hline 9.9 & 9.9 \\
				\hline 9.5 & 9.9 \\
				\hline 9.9 & 9.9 \\
				\hline 9.9 & 9.8 \\
				\hline \hline
					  9.8 & 9.875 \\
			\end{tabular}
		\end{center}
		Измерим период колебаний магнитного стержня:
		\begin{center}
			\begin{tabular}{c|c|c}
				$t$, c & $n$ & $T$,c \\ \hline
				155.53 & 25 & 6.22 \\ \hline
				163.41 & 25 & 6.13 \\ \hline
				155.53 & 25 & 6.22 \\ \hline	
			\end{tabular}
		\end{center}
		По формуле (7) рассчитаем $B_0$.
		\begin{equation}
			B_0 = \frac{2\pi}{TR}\sqrt{\frac{\mu_0JL}{2\pi Rx_1}} = \frac{2\pi}{6.32\cdot 0.2}\sqrt{\frac{1.26\cdot 10^{-6}\cdot 8.24*\cdot 10^{-7}\cdot 1.045}{2\pi \cdot 0.2\cdot 0.098}} = \left(1.47 \pm 0.04\right)\cdot 10^{-5}\,\text{Тл}
		\end{equation}
		\subsection{Определение электродинамической постоянной}
		Измерения проводились при напряжении $U = 99\,\text{В}=0.33\,\text{ед. СГС}$, $n = 50\,\text{Гц}$, $N = 44\,\text{витка}$.
		Проведем измерения отклонения кольца:
		\begin{center}
			\begin{tabular}{c|c}
				$x_{2+}$, см & $x_{2-}$, см \\
				\hline 10.1 & 10.9 \\
				\hline 10.3 & 11 \\
				\hline 10.2 & 10.6 \\
				\hline 10.7 & 11 \\
				\hline 10.9 & 11 \\
				\hline 10.7 & 11.2 \\
				\hline 10.8 & 10.7 \\
				\hline 11.3 & 11 \\
				\hline 10.8 & 11.8 \\
				\hline 10.8 & 11.7 \\ \hline\hline
				 10.66 & 11.09
				
			\end{tabular}
		\end{center}
			Рассчитаем токи:
			\begin{equation}
				I_{\text{СИ}} = \frac{2B_0 R}{\mu_0 N}\tan \phi_2 = \frac{2B_0 R}{\mu_0 N}\frac{x_2}{2L} = \frac{1.47\cdot 10^{-5} \cdot 0.2}{1.26\cdot 10^{-6}\cdot 44} \frac{0.109}{2\cdot 1.045} = 0.55 \pm 0.03 \, \text{А}
			\end{equation}
			\begin{equation}
				I_{\text{СГС}} = CUn = 9\cdot 10^5 \cdot 0.33 \cdot 50 = \left(14.85\pm 0.30\right)\cdot 10^6 \, \text{ед. тока. СГС}
			\end{equation}
			Рассчитаем электродинамическую постоянную:
			\begin{equation}
				c = 10\frac{I_{\text{СГС}}}{I_{\text{СИ}}} = 10\frac{14.85\cdot 10^6}{0.55} = \left(2.70\pm 0.16\right)\cdot 10^8
			\end{equation}
	\section{Вывод}
		В ходе экспериментов нами было получно значение горизонтальной составляющей магнитного поля. С его помощью можно получить выражения для электродинамической постоянной. Оно получилось несколько меньше эталонного. Это произошло потому, что в магнитное поле Земли измерялось вблизи массивных металлических предметов и внутри здания с металлической арматурой, что внесло некоторые искажения.
		
\end{document}


