\documentclass[a4paper,12pt]{article}
\usepackage{amsmath,amssymb,amsfonts,amsthm}
\usepackage{tikz}
\usepackage [utf8x] {inputenc}
\usepackage [T2A] {fontenc} 
\usepackage[russian]{babel}
\usepackage{cmap} 
\usepackage{ gensymb }
% Так ссылки в PDF будут активны
\usepackage[unicode]{hyperref}
\usepackage{ textcomp }

% вы сможете вставлять картинки командой \includegraphics[width=0.7\textwidth]{ИМЯ ФАЙЛА}
% получается подключать, как минимум, файлы .pdf, .jpg, .png.
\usepackage{graphicx}
% Если вы хотите явно указать поля:
\usepackage[margin=1in]{geometry}
% Или если вы хотите задать поля менее явно (чем больше DIV, тем больше места под текст):
% \usepackage[DIV=10]{typearea}

\usepackage{fancyhdr}

\newcommand{\bbR}{\mathbb R}%теперь вместо длинной команды \mathbb R (множество вещественных чисел) можно писать короткую запись \bbR. Вместо \bbR вы можете вписать любую строчку букв, которая начинается с '\'.
\newcommand{\eps}{\varepsilon}
\newcommand{\bbN}{\mathbb N}
\newcommand{\dif}{\mathrm{d}}

\newtheorem{Def}{Определение}


\pagestyle{fancy}
\makeatletter % сделать "@" "буквой", а не "спецсимволом" - можно использовать "служебные" команды, содержащие @ в названии
\fancyhead[L]{\footnotesize Оптика}%Это будет написано вверху страницы слева
\fancyhead[R]{\footnotesize ФМХФ МФТИ}
\fancyfoot[L]{\footnotesize \@author}%имя автора будет написано внизу страницы слева
\fancyfoot[R]{\thepage}%номер страницы —- внизу справа
\fancyfoot[C]{}%по центру внизу страницы пусто

\renewcommand{\maketitle}{%
	\noindent{\bfseries\scshape\large\@title\ \mdseries\upshape}\par
	\noindent {\large\itshape\@author}
	\vskip 2ex}
\makeatother
\def\dd#1#2{\frac{\partial#1}{\partial#2}}


\title{4.7.3 \\ Поляризация}
\author{Егор Берсенев} 
\date{16 февраля 2017 г.}

\begin{document}
	\maketitle
	\section{Цель работы}
		Ознакомление с методами получения и анализа поляризованного света.
	\section{Оборудование}
	Оптическая скамья с осветителем; зеленый светофильтр;
	два поляроида; черное зеркало; полированная эбонитовая пластинка; стопа стеклянных пластинок;
	слюдяные пластинки разной толщины; пластинки в 1/4 и 1/2 длины волны; пластинка в одну
	волну для зеленого света (пластинка чувствительного оттенка).
	
	В естественном свете ориентация векторов $\vec E$ и $\vec H$ в плоскости,
	перпендикулярной вектору $\vec S$ меняется хаотически. Все направления 
	- равноправны.
	
	При помощи поляризаторов естественный свет может быть превращен в линейно 
	поляризованный, то есть такой в котором пара векторов $\vec E$ и $\vec H$
	не меняет со временем своей ориентации.
	
	Наиболее общий тип поляризации - эллиптическая поляризация. В этом случае
	конец вектора $\vec E$ описывает эллипс. Если спроектировать его на два
	взаимно перпендикулярных направления, то получим
	\begin{equation}
		E_x=E_{x0}\cos\omega t,\quad E_y=E_{y0}\cos(\omega t+\varphi)   
	\end{equation}
	
	При прохождении света через поляризатор справедлива 
	следующая формула (\it закон Малюса\rm):
	\begin{equation}
		I=I_0\cos^2\alpha
	\end{equation}
	где $\alpha$ - угол между разрешенной плоскостью и 
	плоскостью колебаний.
	
	\section{Ход работы}
	
		\subsection{Определение разрешенной плоскости колебаний}
		
		Соберем оптическую схему, состоящую из осветителя, поляроида и черного зеркала.	Будем поворачивать поляроид, находя минимальную яркость отражения при угле зеркала в $45\degree$.
		Затем будем поворачивать зеркало, опять же, добиваясь минимальной
		яркости. Уточним положение поляроида. Показание на лимбе поляроида \textnumero 1 --- 3\degree, на лимбе поляроида  \textnumero 1 --- 36\degree,
		
		\subsection{Исследование угла Брюстера}
		
		Заменим черное зеркало на эбонитовую пластинку. Выставляя пластинку перпендикулярно определим угол на лимбе. Он равен $\alpha_0 = 174\pm0.5$\degree. Угол, при котором яркость минимальна равен  $\alpha_1 = 232\pm0.5$\degree $\implies\Delta\alpha = 58\pm1$\degree $\implies n = \tan\Delta\alpha = 1.6\pm 0.07$
		
		Теперь установим зеленый светофильтр и повторим измерение с ним. 
		Угол, при котором яркость минимальна равен  $\alpha_1 = 233\pm0.5$\degree $\implies\Delta\alpha = 59\pm1$\degree $\implies n = \tan\Delta\alpha = 1.66\pm 0.07$
		
		\subsection{Исследование стопы. Определение направления вектора $\vec{\bf{E}}$}
		
			Теперь заменим эбонитовую пластинку на стеклянную стопу. Также найдем угол Брюстера. Освещая стопу неполяризованным светом определим направление вектора $\vec{\mathbf{E}}$. Поскольку колебания электрического вектора лежат в плоскости, перпендикулярной плоскости падения, то с помощью поляроидов легко находим его направление. 
		
		Наблюдая прошедший через стопу стеклянных пластинок луч света,
			убеждаемся, в том что плоскости поляризации у отраженного
			и преломленного лучей взаимно перпендикулярны. 
	
		\subsection{Определение главных направлений в двоякопреломляющих пластинках. Отбор $\lambda/4$ и $\lambda/2$}
		Пластинка в четверть длины волны создает сдвиг фаз на $\frac{\pi}{2}$, тем самым обеспечивая круговую поляризацию. На опыте это можно наблюдать следующим образом: если при вращении второго поляризатора интенсивность света не меняется, то мы наблюдаем именно круговую поляризацию. В случае с пластинкой в половину длины волны происходит лишь поворот плоскости колебаний электрического вектора с переходом в другой квадрант, тип поляризации при этом, однако, не меняется. Следовательно, наш свет все еще линейно поляризован, а значит, мы можем найти такое положения поляроида, при котором свет будет практически полностью затемнен.
		
		
		\subsection{Определение быстрого и медленного направления}
		С помощью пластинки чувствительного оттенка определим у
		пластинки в $\lambda/4$ главные направления, соответствующие большей и
		меньшей скорости распространения света. Установим между скрещенными 
	поляроидам последовательно пластинки в одну длину волны и в половину длины волны.
	Зеленый светофильтр уберем. Повернем пластинку в $\lambda/4$ до тех пор пока
	прошедший свет не станет зеленовато-голубым. Тогда главное направление,
	соответствующее большей скорости будет совпадать со стрелкой.
	
	\subsection{Исследование интерференции поляризованных лучей}
	Поместим между поляроидами пластинку, собранную из листков слюды. 
		Проведем наблюдение в следующих двух случаях: 
		а) пластинка поворачивается
		между скрещенными поляроидами. В этом случае меняется яркость света, так как меняется поляризация.
		б) пластинка неподвижна, поворачивается 
		анализатор. В этом случается меняется цвет, поскольку происходит сдвиг фаз.
		Оба явления проявляются вследствие интерференции поляризованных лучей.

	\section{Вывод}
	Поляризованный свет обладает большим числом свойств, которые можно применять для исследования оптических характеристик различных приборов и веществ.
	
\end{document}


