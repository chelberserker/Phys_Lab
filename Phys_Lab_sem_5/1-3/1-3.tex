\documentclass[a4paper,12pt]{article}
\usepackage{amsmath,amssymb,amsfonts,amsthm}
\usepackage{tikz}
\usepackage [utf8x] {inputenc}
\usepackage [T2A] {fontenc} 
\usepackage[russian]{babel}
\usepackage{cmap} 
\usepackage{ gensymb }
% Так ссылки в PDF будут активны
\usepackage[unicode]{hyperref}
\usepackage{ textcomp }
\usepackage{indentfirst}
\usepackage[version=3]{mhchem}

% вы сможете вставлять картинки командой \includegraphics[width=0.7\textwidth]{ИМЯ ФАЙЛА}
% получается подключать, как минимум, файлы .pdf, .jpg, .png.
\usepackage{graphicx}
% Если вы хотите явно указать поля:
\usepackage[margin=1in]{geometry}
% Или если вы хотите задать поля менее явно (чем больше DIV, тем больше места под текст):
% \usepackage[DIV=10]{typearea}

\usepackage{fancyhdr}

\newcommand{\bbR}{\mathbb R}%теперь вместо длинной команды \mathbb R (множество вещественных чисел) можно писать короткую запись \bbR. Вместо \bbR вы можете вписать любую строчку букв, которая начинается с '\'.
\newcommand{\eps}{\varepsilon}
\newcommand{\bbN}{\mathbb N}
\newcommand{\dif}{\mathrm{d}}

\newtheorem{Def}{Определение}


\pagestyle{fancy}
\makeatletter % сделать "@" "буквой", а не "спецсимволом" - можно использовать "служебные" команды, содержащие @ в названии
\fancyhead[L]{\footnotesize Квантовая физика}%Это будет написано вверху страницы слева
\fancyhead[R]{\footnotesize ФМХФ МФТИ}
\fancyfoot[L]{\footnotesize \@author}%имя автора будет написано внизу страницы слева
\fancyfoot[R]{\thepage}%номер страницы —- внизу справа
\fancyfoot[C]{}%по центру внизу страницы пусто

\renewcommand{\maketitle}{%
	\noindent{\bfseries\scshape\large\@title\ \mdseries\upshape}\par
	\noindent {\large\itshape\@author}
	\vskip 2ex}
\makeatother
\def\dd#1#2{\frac{\partial#1}{\partial#2}}


\title{1.3 \\ Эффект Рамзауэра}
\author{Егор Берсенев} 
\date{16 февраля 2017 г.}

\begin{document}
	
	\maketitle
	\section{Теоретическое введение
		}
	Эффективное сечение реакции --- это величина, характеризующая вероятность перехода системы двух сталкивающихся частиц в результате их рассеяния (упругого или неупругого) в определенное конечное состояние. Сечение $\sigma$ это отношение числа таких переходов $N$ в единицу времени к плотности потока $nv$ рассеиваемых частиц, падающих на мишень, т.е. к числу частиц, попадающих в единицу времени на единичную площадку, перпендикулярную к их скорости.
	
	\begin{equation}
		\sigma = \frac{N}{nv}
	\end{equation}
	
	Эффект Рамзауэра нельзя объяснить с позиций классической теории. С квантовой же точки зрения картина рассеяния выглядит следующим образом. Внутри атома потенциальная энергия налетающего электрона отлична от нуля, скорость электрона меняется, становясь равной $v'$ в соответсвии с законом сохранения энергии:
	\begin{equation}
		E = \frac{mv^2}{2} = \frac{mv'^2}{2} + U
	\end{equation}
	
	а значит, изменяется и длина его волны де Бройля. Таким образом, по отношению к электронной волне атом ведет себя как преломляющая среда с относительным показателем преломления:
	
	\begin{equation}
		n = \frac{\lambda}{\lambda'} = \sqrt{1 - \frac{U}{E}}
	\end{equation}
	
	Решение задачи о рассеянии электрона на сферическом потенциале достаточно громоздко. Поэтому рассмотрим более простое одномерное приближение: электрон рассеивается на потенциальной яме конечной глубины. Уравнение Шрёдингера в этом случае имеет вид:
	\begin{equation}
		\psi'' + k^2\psi = 0 \qquad k^2 = \begin{cases}
		 k_1^2  = \frac{2mE}{\hbar^2} \\
		 k_2 = \frac{2m(E+U_0)}{\hbar^2}
		\end{cases}
	\end{equation}
	
	Коэффициент прохождения равен отношению квадратов амплитуд прошедшей и падающей волн и определяется выражением:
	\begin{equation}
		D = \frac{16k_1^2k_2^2}{16k_1^2k_2^2 + 4(k_1^2-k_2^2)^2\sin^2(k_2l)}
	\end{equation}
	
	Видно, что коэффициент прохождения частицы над ямой, в зависимости от её энергии, имеет вид чередующихся максимумов и минимумов. В частности, если $k_2l = \pi$, то коэффициент прохождения равен 1, т.е. отраженная волна отсутствует, и электрон беспрепятственно проходит через атом. Этот эффект является квантовым аналогом просветления оптики. Таким образом, коэффициент прохождения электронов максимален при условии:
	\begin{equation}
		k_2l = \sqrt{\frac{2m(E+U_0)}{\hbar^2}}l = \pi n
	\end{equation}
	Прошедшая волна 1 усилится волной 2, если геометрическая разность хода между ними $\Delta = 2l = \lambda'$, что соответствует условию первого интерференционного максимума, т.е.
	\begin{equation}
		2l = \frac{h}{\sqrt{2m(E_1 + U_0)}}
	\end{equation}
	C другой стороны, прошедшая волна ослабится, если $2l = \frac{3}{2}\lambda'$, т.е.
	\begin{equation}
		2l = \frac{3}{2}\frac{h}{\sqrt{2m(E_2+U_0)}}
	\end{equation}
	Решая эти уравнения совместно можно исключить $U_0$ и найти эффективный размер атома $l$:
		\begin{equation}
			l = \frac{h\sqrt{5}}{\sqrt{2m(E_2-E_1)}}
		\end{equation}
	Понятно, что энергии $E_1$, $E_2$ соответсвуют энергия электронов, прошедших разность потенциалов $V_1$ и $V_2$.
	Кроме того, можно оценить эффективную глубину потенциальной ямы атома:
	\begin{equation}
		U_0 =\frac{4}{5}E_2 - \frac{9}{5}E_1
	\end{equation}
	
	Теперь рассмотрим ВАХ тиратрона. Она имеет вид:
	$$
	I_a = I_0e^{-C\omega(V)}, C = Ln_a\Delta_a
	$$
	где $I_0 = eN_0$ --- ток катода, $I_a = eN_a$ --- анодный ток, $\Delta_a$ --- площадь поперечного сечения атома, $n_a$ --- концентрация атомов газа в лампе, $L$ --- расстояние от катода до анода, $\omega(V)$ --- вероятность рассеяния электрона на атоме как функция от ускоряющего напряжения. По измеренной ВАХ тиратрона можно определить зависимость вероятности рассеяния электрона от его энергии из соотношения:
	\begin{equation}
		\omega(V) = -\frac{1}{C}\ln\frac{I_a}{I_0}
	\end{equation}
	
	\section{Экспериментальная часть}
		Проведем измерения в динамическом режиме:
		\begin{figure}[h!]
			\centering
			\includegraphics[width=0.4\linewidth]{pic1}
			\caption{Вид ВАХ в динамическом режиме}
		\end{figure}
		
		\begin{table}[h!]
			\centering
			\label{my-label}
			\begin{tabular}{|l|l|l|}
				\multicolumn{3}{l}{$V_g = 2.645 \pm 0.1 V $} \\ \hline
				& $V_{max}$ & $V_{min}$ \\ \hline
				верхняя кривая    &  2  & 6    \\ \hline
				нижняя кривая  &  2  &  5.5       \\ \hline
			\end{tabular}
		\end{table}
		
		По ВАХ оценим напряжение пробоя. Оно равно примерно $V \simeq 10 V$. Следовательно, наш газ --- ксенон.
		
		Теперь перейдем к измерениям в статическом режиме. Снимем ВАХ для двух значений напряжения накала.
		
		\begin{figure}[h!]
			\centering
			\includegraphics[width=0.7\linewidth]{graph1}
		\end{figure}
		
		\begin{figure}[h!]
			\centering
			\includegraphics[width=0.7\linewidth]{graph2}
		\end{figure}
		
		По формулам рассчитаем характерный размер электронной оболочки атома ксенона и глубину потенциальной ямы.
		
		\begin{table}[h!]
			\centering
			\label{my-label}
			\begin{tabular}{|l|l|l|} \hline
				& 1     & 2     \\ \hline
				$U_g, V$   & 2.645 & 2.82  \\ \hline
				$l, \,$\AA & $3.1\pm0.1$  & $3.1\pm0.1$  \\ \hline
				$U_0, eV$  & $2.242\pm0.2$ & $2.082\pm0.2$ \\ \hline
			\end{tabular}
		\end{table}
		
		Найдем зависимость энергий, соответствующих максимум коэффициента прохождения электронов $E_n = f(E_1, n)$:
		$$
			E_n = n^2\left(E_1+U_0\right)-U_0 \implies \begin{cases}
			E_2 = 13.4\pm0.2 eV \\
			E_3 = 32.8\pm0.2 eV
			\end{cases}
		$$ 
		
		На основе формулы для вероятности рассеяния электрона построим график этой вероятности в зависимости от анодного напряжения. Константу C выберем из соображений красивого качественного графика.
		
		\begin{figure}[h!]
			\centering
			\includegraphics[width=0.7\linewidth]{omega}
		\end{figure}
		
		\section{Обсуждение результатов и выводы}
			В проделанной работе было изучено явление рассеяния электронов на атомах ксенона. Экспериментальные данные подтверждают гипотезу о волновых свойствах электрона.Были оценены размеры электронной оболочки ксенона и глубина потенциальной ямы атома. Кроме того, было исследовано влияние магнитного поля на ВАХ тиратрона. В ходе этого исследования был выявлен интересный эффект, последовательного увеличения горба, а затем уменьшения. Объяснить данный эффект не удалось.
			
\end{document}


